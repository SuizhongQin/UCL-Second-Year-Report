\documentclass[12pt]{article}
\usepackage[margin= 1in]{geometry}
\usepackage[utf8]{inputenc}
\usepackage{graphicx}
\usepackage{float}
\usepackage{listings}
\usepackage{amsmath}
\usepackage[justification=centering]{caption}
\usepackage[nottoc]{tocbibind}
\usepackage{fancyhdr}
\usepackage{wrapfig}

%footer
\lfoot{}
\cfoot{}
\rfoot{}

\makeatletter % `@' now normal "letter"
\@addtoreset{equation}{section}
\makeatother  % `@' is restored as "non-letter"
\renewcommand\theequation{\oldstylenums{\thesection}%
                   .\oldstylenums{\arabic{equation}}}

%title
\title{Scenario W: EMG Sensor \\ Technical Report}
\author{Student ID: 17018962 \\ \\ Department: Electronic and Electrical Engineering \\ University College London}

\lstset{breaklines}
\lstset{extendedchars=false}
 
 
%%%%%%%%%%%% Document Beginning %%%%%%%%% 
\begin{document}
    
    \begin{figure}[t]
    \centering
    \includegraphics[width=0.8\textwidth]{logo.pdf}
    \end{figure}
    \maketitle
    \thispagestyle{empty}

    \newpage
    \tableofcontents
    \thispagestyle{empty}
    \setcounter{page}{0}
    
    \newpage
    \pagestyle{plain}
    \section{Introduction}
    \paragraph{}
    Along with the deepening of people's cognition of Parkinson's disease, motor neuron disease, and muscular atrophy. The research of nerves and muscles becomes a popular topic in modern biological and medical fields. Patients with relative myopathies usually perform a malfunction of muscular movement or control, such as muscle tremors, rigidity, and slowness of movement. In order to investigate the nature of this kind of disease, former scientists found that electrical activities could be generated by the contraction of muscles. Later, the EMG(Electromyography) was introduced, which is a new electrodiagnostic medicine technique for evaluating and recording the electrical signals caused by skeletal muscles\cite{definition}. By connecting the surface of human skin, a EMG sensor equipment could help doctors diagnose muscular illness more accurately by observing shapes of signals.
    
    \paragraph{}
    During the Scenario W week, we were expected to build a simple EMG sensor system. To be specific, collecting the small signals from muscles, and processing them to obtain optimum values. The output should be displayed by audio and visual user interfaces to demonstrate the strength of muscles. In this technical report, the whole design, simulation, building, and testing processes will be summarised. 
    
    \section{Background}
    \subsection{Literature Review}
    \paragraph{}
    Early research in EMG started from  Francesco Redi’s first discovery in 1666, a high voltage electricity could be generated by a electric eel's muscle. Later, Aloysii Galvani found that electricity could trigger the muscle to contract in his publication  De Viribus Electricitatis in 1792\cite{1}. About one century passed, the term EMG (Electromyography) was introduced by Marey. He was the first human recorded EMG activity. In the 1920s, only very distorted and vague signals could be observed on an oscilloscope by Gasser and Erlanger. In the following 30 years, the technique enhanced significantly which resulted a pervasive of the surface EMG. With the rapid development in electrodes' integrated circuits in the 1980s, this led a prosperous commercial production of the EMG. At the same time, a new concept was introduced by Cram and Steger, the EMG signal could be detected by scanning muscles\cite{2}. In recent research, more surface EMG is used to record the muscle's superficial layer, and intramuscular electrodes are used to evaluate the deeper layer of the muscle's activity\cite{3}.  
    
    
    \subsection{Theory}
    \paragraph{}
    \begin{figure}[H]
            \centering
            \includegraphics[width=0.6\textwidth]{system.png}
            \caption{The Main Systems of the EMG Sensor}
        \end{figure}
    This Scenario Week was aimed to build a surface EMG and measure the signal from the skin. The four main systems are shown in Figure 1. A 100 mV electrical signal would be produced when human's brain orders the muscle to contract. However, this signal would be weakened by the internal body tissue and the skin. When the original signal arrived at the surface of the skin, the signal's energy was dissipated greatly, the EMG signal detected from the skin was quite small. The typical value for EMG frequency range is from 10 Hz to a few kilo Hertz, and the amplitude range is from Sub-mV to a few mV. (e.g. When measuring the biceps, the EMG amplitude is about 1-2 mV.) In order to obtain a readable value for Arduino(0-5V), a amplifier is needed to amplify the signal. Due to noises from internal body tissues, a filter are required to add in the circuit to eliminate the unwanted part.  After that, the signal in the negative part could be inverted by a full-wave rectifier. The final stage is to smooth and average the signal by designing a integrator circuit. Connecting the processed signal as the input of the Arduino Board. By programming on the Arduino IDE, the final output could be demonstrated by using a LED Bar-Graph Display and a Speaker.
   
    \section{Method and Results}
    \subsection{EMG Electrodes}
    \paragraph{}
    \begin{wrapfigure}{c}{2.5cm}
    \includegraphics[width=2.5cm]{electrodes.jpg}
    \caption{Electrodes Made by Coin}
    \end{wrapfigure}
    Electrodes are used to transmit signal from the skin to the EMG sensor. To get the best results, coins are selected as the material to make EMG Electrodes. Considering the electrical conductivity for different coins, 2 Pence is chosen as it contains 97\% copper\cite{4}. In order to cancel the common mode signal, one extra electrode is required to act as a reference point. Thus, a total three coins need to be soldered with conducting wires. The conducting side of coins should be polished by the sand paper. Make sure the conductive adhesive gel is spread evenly before connecting with the skin as well. These are aimed to minimise the noise from coins.
    
    
    \subsection{Instrumentation Amplifier}
    \paragraph{}
    \begin{figure}[H]
            \centering
            \includegraphics[width=0.6\textwidth]{IA.png}
            \caption{Instrumentation Amplifier and Driven Right Leg Circuit}
        \end{figure}
    As shown in Figure 3, V1 and V2 are the two inputs from electrodes. This could result a common mode signal and influence the measured result. To eliminate the common mode signal, a driven right leg circuit is included. For the sake of budget, TL084ACD is chosen as the component of driven right leg circuit. This part of circuit does not need high input impedance as well. As for the design of the amplifier, the signal detected from the skin is below 10 mV. This is too small to obtain on the Arduino Board. Considering the full-wave rectifier in the later circuit, an Instrumentation Amplifier with 800 gain is required to ensure the output value is between the range 0-5V. The gain of the Instrumentation Amplifier is:
    \begin{equation}
    Gain = \frac{R_4}{R_3}\times \left ( 1+\frac{R_2}{R_1} \right )
    \end{equation}
    Inserting values, $R_1=500\Omega$, $R_2=20k\Omega$,$R_3=1k\Omega$,and $R_4=20k\Omega$ into the (3.1), then $Gain=820$ which is satisfy the designing gain value. For the high input impedance, the Instrumentation Amplifier would be the best option.
    
    
    \subsection{Signal Processing}
    \subsubsection{High-pass Filter}
    \paragraph{}
    \begin{wrapfigure}{c}{2.5cm}
    \includegraphics[width=2.5cm]{highpass.png}
    \caption{High-pass Filter}
    \end{wrapfigure}
    The amplified signal will flow into the High-pass Filter. Although common mode signals are removed by the Driven Right Leg, there is still existing noise, which is the  DC offset comes from the skin. The DC offset will be amplified and influence the result. Therefore, a High-pass Filter is designed to eliminate this noise. The EMG frequency range is from 10 Hz to a few kilo Hertz, the designing values for $R13=R14=100k\Omega$, and $C_1=0.1\mu F$. The cut-off frequency is:
    \begin{equation}
                    f_c = \frac{1}{2 \pi R C}
    \end{equation}
    Inserting values of $R$ and $C$, cut-off frequency is equal to 15.9 Hz.
    
    \subsubsection{Rectification}
    \paragraph{}
    \begin{figure}[H]
            \centering
            \includegraphics[width=0.2\textwidth]{rectifier.png}
            \caption{Rectifier Circuit}
        \end{figure}
    The filtered signal will then flow into the rectifier. Since the final output is a continuous and positive signal, a full-wave rectifier(Figure 5) must be used to invert the negative part of the signal. A Superdiode circuit acts as the rectifier in this system. When the signal is positive, the upper diode is forward-biasing, and the lower diode is reverse-biasing. The output can be deduced:
    \begin{equation}
                   V_i=\left ( -\frac{R}{R} \right )\left ( -\frac{R}{R} \right )V_o=V_o
    \end{equation}
    When the signal is negative, the upper diode is reverse-biasing, and the lower diode is forward-biasing. Using Kirchhoff's Current Law:
    \begin{equation}
                  \frac{V_i}{R}+\frac{V}{2R}+\frac{V}{R}=0
    \end{equation}
    By rearranging the equation, $V=-\frac{2}{3}V_i$. The output is:
    \begin{equation}
                 V_0=\left ( 1+\frac{R}{2R} \right )V=\frac{3}{2}V=\frac{3}{2}\times \frac{-2}{3}V_i=-V_i
    \end{equation}
    
    
    
    \subsubsection{Integrator and Inverted Amplifier}
    \paragraph{}
    \begin{wrapfigure}{c}{5cm}
    \includegraphics[width=5cm]{final.png}
    \caption{Integrator and Inverted Amplifier}
    \end{wrapfigure}
    After the signal is rectified, it flows into the integrator. This part of the circuit is aimed to smooth the signal and get an average value.As shown on Figure 6, the designing value of gain is -1 and cut-off frequency is 1.975Hz. The integrator is a low-pass filter, and the gain is negative, which means another inverted amplifier is needed to ensure the final output is positive. After the signal flow through the integrator, then it will flow into the inverted amplifier with -20 gain. This is to match the final output with the Arduino input range (0-5V). 
    
    
    
    
    \subsection{Output}
    \paragraph{}
    \begin{figure}[h]
    \centering
    \begin{minipage}[t]{0.48\textwidth}
    \centering
    \includegraphics[width=6cm]{led.png}
    \caption{Arduino Circuit for the LED Bar-Graph}
    \end{minipage}
    \begin{minipage}[t]{0.48\textwidth}
    \centering
    \includegraphics[width=6cm]{speaker.png}
    \caption{Arduino Circuit for the Speaker}
    \end{minipage}
    \end{figure}
    \paragraph{}
    LED bar and speaker as the user interfaces to demonstrate the final output. As for the LED bar, ten individual LEDs is given to demonstrate the strength of muscles. By using the 'map' function in Arduino IDE, more LEDs will light up as the EMG signal's amplitude increases. Connecting a 220 $\Omega$ resistor to each LED in series is to limit current and protect the circuit. As for the Speaker, using the 'tune' function in Arduino IDE to allocate Serial Read into appropriate intervals of sound frequencies. The sound frequency rises as the muscle strength increases.
    

    
    \section{Discussion and Conclusion}
    \subsection{Improvements}
    \begin{itemize}
   \item Use professional electrodes instead of coins, this could remove noises and poor contact with the skin.   
   \item Use the Instrumentation Amplifier with larger gain. Due to the instability of electronic components, the real values may be different from nominal values. This could cause a fail(very low signal received by Arduino) during the test.
   \item Try to solder all the component on a Strip Board if time is sufficient. This would increase the stability of the electric circuit.
   \end{itemize}
    
    
    
    \subsection{Conclusion and Future Research}
    \paragraph{}
    This project mainly focus on the amplification of a small electrical signal to a desired value. By building the EMG Sensor, the theory is designed to implement into practice. Obtaining the signal from the skin, then the signal flows through Instrumentation Amplifier, High-pass filter, Rectifier, Integrator, and Inverted Amplifier. Finally demonstrating by a LED Bar and a speaker. Even though the simulation on MultiSim always works, the real circuit often malfunctions. By considering the design defects and external factors, there are several improvements that could increase the stability of the system.
    \paragraph{}
    EMG would be a popular research area in the future. A wireless EMG Sensor may be introduced in the future. In order to implement this challenge, a dramatic development in wireless transmission technology is required. It could be used not only for diagnosis of deceases, but also for athletic training. For the sports like sprint, swimming, and weightlifting, the wireless EMG Sensor allows them record and evaluate their muscles' performance without the restriction of distance and surroundings. The EMG Sensor helps people understand their muscles' power and get better results.   
    
     
    \newpage   
    \begin{thebibliography}{}
        \bibitem{definition}
        En.wikipedia.org. (2019). Electromyography. [online] Available at: https://en.wikipedia.org/wiki/Electromyography [Accessed 27 Nov. 2019].

        \bibitem{1}
        A Translation of Luigi Galvani's De viribus electricitatis in motu musculari commentarius. Commentary on the Effect of Electricity on Muscular Motion. (1953). Journal of the American Medical Association, 153(10), p.989.
        
        \bibitem{2}
        Cram, J. and Steger, J. (1983). EMG scanning in the diagnosis of chronic pain. Biofeedback and Self-Regulation, 8(2), pp.229-241.
        
        \bibitem{3}
        En.wikipedia.org. (2019). Electromyography. [online] Available at: https://en.wikipedia.org/wiki/Electromyography [Accessed 27 Nov. 2019].
        
        \bibitem{4}
        Compound Interest. (2019). The Metals in UK Coins | Compound Interest. [online] Available at: https://www.compoundchem.com/2014/03/27/the-metals-in-uk-coins/ [Accessed 28 Nov. 2019].
    \end{thebibliography}
            
    
\end{document}



% definition: https://en.wikipedia.org/wiki/Electromyography